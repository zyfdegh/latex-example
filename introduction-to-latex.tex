\documentclass{article}

\title{Introduction to LaTeX}

\begin{document}

\section{Introduction}
This file shows how to write an article using LaTeX.

LaTeX is a language for writing document, similar as Markdown. With sections, tables, labels,
references, math expressions. Of course, it is known as a geek editor for mathematics.

This article is totally written with LaTeX.

\section{Install}

Any text editor can be used to write LaTeX files.

For macOS, MacTex is recommended, but it's hugo and up to 3.4GB. 

Personally, I'm using Texpad which have a realtime PDF preview of your document.

The LaTeX file extension is *.tex.

\section{Sytax}
In this section, I will show you how to using the syntax.

\subsection{Tables}

Difficulty comparison of LaTeX and Markdown

\begin{tabular}{r|r|r|r|r|r}
\hline
Name & Section & Table & Link & Media & Math\\
\hline
LaTeX & easy & medium & - & - & - \\
\hline
Markdown & easy & easy & easy & - & - \\
\hline
\end{tabular}

\subsection{Lists}\label{section-lists}

Unordered list

\begin{itemize}
	\item Atom
	\item vscode
	\item Vim
\end{itemize}

Ordered list

\begin{enumerate}
	\item First, learn about the syntax
	\item Then try yourself with an editor
	\item Publish the article
\end{enumerate}

\subsection{Labels and references}

If you want to reference a section in the article, we can use labels.

For example if you want to refer "3.2 List" here, you need to first set label 
for the section with label{labelname}, then use ref{labelname} for the reference.

Refer to \ref{section-lists}

\subsection{Math}

Use one pair of dollor sign to wrap inline Math expressions. For example, $E=mc^2$

Use two pairs of dollor sign to display math formula in a new line and centered.


\subsection{Examples}

From WikiPedia "Quantum mechanics"

$$E=h\nu$$

$$-\frac{\hbar^2}{2m}\frac{d^2 \psi}{dx^2}=E\psi$$

$$\hat{p}_x=-i\hbar\frac{d}{dx}$$

$$\frac{1}{2m}\hat{p}_x^2=E$$

$$\psi(x)=Ae^{ikx}+Be^{-ikx}$$
$$E=\frac{\hbar^2 k^2}{2m}$$

$$\psi(x)=C\sin kx+D\cos kx$$
$$\psi(0)=0=C\sin 0+D\cos 0=D$$

$$\psi(L)=0=C\sin kL$$

$k=\frac{n\pi}{L}$ \qquad $n=1,2,3,....$


$$E=\frac{\hbar^2 \pi^2 n^2}{2mL^2}=\frac{n^2h^2}{8mL^2}$$

$$V(x)=\frac{1}{2}m\omega^2x^2$$

$$\psi(x)=\sqrt{\frac{1}{2^n n!}}\cdot\left( \frac{m\omega}{\pi\hbar} \right)^{1/4}\cdot e^{-\frac{m\omega x^2}{2\hbar}}\cdot H_n \left( \sqrt{\frac{m\omega}{\hbar}x} \right)$$

\section{References}
[1] David Xiao.A beginner's Guide to LaTeX. https://www.cs.princeton.edu/courses/archive/spr10/cos433/Latex/latex-guide.pdf

[2] LaTeX Mathematical Symbols. https://reu.dimacs.rutgers.edu/Symbols.pdf


\end{document}